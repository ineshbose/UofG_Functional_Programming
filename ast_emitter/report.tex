\documentclass{article}
\usepackage[utf8]{inputenc}

\title{%
    Functional Programming (H)\\
    AST Pretty Printer
}
\author{Inesh Bose}
\date{}

\begin{document}

\maketitle

\section*{Design}

It is understood that all the data type syntax relies on \texttt{Expr}. In order to allow the program to compile, first the higher-order types \texttt{zipt}, \texttt{unzipt}, \texttt{map}, and \texttt{stencil} will be added referring to the provided example:

\begin{verbatim}
    map :: (a -> b) -> Vec k a -> Vec k b
    zipt :: (Vec k a,Vec k b,...) -> Vec k (a,b,...)
    unzipt :: Vec k (a,b,...) -> (Vec k a,Vec k b,...)
    stencil :: SVec sk Int -> Vec k a -> Vec k (SVec sk a))
\end{verbatim}

\noindent The pretty-printer for an AST instance can then be created using:

\begin{itemize}
    \item the given pre-defined functions \texttt{ppProgram}, \texttt{ppBindings}, \texttt{ppAST}, \texttt{ppExprTup}, \texttt{ppDType}, \texttt{ppMainTypeDecl}, \texttt{ppMainArgDef}, and \texttt{ppMainReturnDef};
    \item define \texttt{ppStencilDef} and \texttt{ppArgDecl} using \texttt{show} (for list of \texttt{Integer}) and \texttt{ppDType} (for \texttt{DType}) respectively ideally in one line;
    \item \texttt{ppFSig} may require a helper function with guards to evaluate \texttt{Expr};
    \item finally, \texttt{ppLHSExpr} and \texttt{ppRHSExpr} that are used for all let-bindings in the main function will be similar to the previous functions, taking advantage of \texttt{ppDType} and \texttt{ppArgName}.
\end{itemize}

\noindent Majority of the development will be driven through the given examples in \texttt{ASTInstances.hs}. Additional tests would be planned to be written.

\end{document}
